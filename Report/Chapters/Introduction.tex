\chapter{Introduction}

\section{Background}
In recent years, the field of hand gesture recognition has gained significant attention due to its applications in areas such as human-computer interaction (HCI), virtual reality (VR), augmented reality (AR), and sign language recognition. One of the primary challenges in this domain is the accurate recognition of hand gestures, particularly when the gestures are static and performed at the character level, which is common in sign languages. \cite{oyedotun2017deep} Traditional methods often struggle to recognize fine-grained distinctions between hand gestures, which makes it essential to explore more advanced and efficient approaches.\\

With the advent of deep learning, pre-trained models trained on large datasets such as ImageNet have proven to be highly effective for various image classification tasks. These models, including architectures like Convolutional Neural Networks (CNNs), Vision Transformers (ViTs), and EfficientNets, have shown significant improvements in image recognition accuracy. Their transfer learning capabilities allow them to perform well in new tasks, such as character-level static hand gesture recognition, with minimal additional training.\\

This research aims to conduct a comparative study of several state-of-the-art pre-trained ImageNet models for character-level static hand gesture recognition. The study will evaluate models such as VITL32, EfficientNetV2L, VGG16, and others based on multiple performance metrics, including precision, precision, recall, sensitivity, and F1 score of Top-1, Top-2 and Top-3. Furthermore, the research will incorporate advanced visualization techniques, such as t-SNE, GradCAM heat maps, and lime visualization, to interpret the model decision-making processes and identify potential areas for improvement.